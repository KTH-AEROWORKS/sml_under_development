
\tikzstyle{startstop} = [rectangle, rounded corners, minimum width=3cm, minimum height=1cm,text centered, draw=black, fill=red!30,text width=3cm]

%\tikzstyle{io} = [trapezium, trapezium left angle=70, trapezium right angle=110, minimum width=3cm, minimum height=1cm, text centered, draw=black, fill=blue!30]
\tikzstyle{io} = [trapezium, trapezium left angle=70, trapezium right angle=110, minimum width=3cm, minimum height=1cm, text centered, draw=black, fill=blue!30,text width=3cm]

%\tikzstyle{process} = [rectangle, minimum width=3cm, minimum height=1cm, text centered, draw=black, fill=orange!30]
\tikzstyle{process} = [rectangle, minimum width=3cm, minimum height=1cm, text centered, text width=7cm, draw=black, fill=orange!30]

\tikzstyle{decision} = [diamond, minimum width=3cm, minimum height=1cm, text centered, draw=black, fill=green!30]
\tikzstyle{arrow} = [thick,->,>=stealth]

\begin{landscape}
	%
	\begin{figure}
		\begin{tikzpicture}[node distance=2cm,font=\footnotesize]
			%
			\node (start) [startstop] {Start \\ Given \textbf{dic}};
			%
			\node (in1) [decision, below of=start] {Reset \\ Button};
			%
			\node (pro1) [process, below of=in1] {print \textbf{dic}.keys() \\ head\_class\_set = \textbf{F}};
			%
			\node (in2) [decision, below of=pro1, yshift=-0.5cm] {Item Clicked};
			%
			\node (pro2a) [process, below of=in2, xshift=-5cm,yshift = -2cm] 
			{	
				\begin{flushleft}
					head\_class\_set = \textbf{T} 
					\\
					head\_class\_key = \textbf{get\_text()}
					\\
					HeadClass = \textbf{dic}(head\_class\_key)
					\\
					input\_dic = HeadClass.inner()
					\\
					if check\_completeness(input\_dic)
					\\
					\quad print  HeadClass.parameters()
					\\
					else
					\\
					\quad list\_keys = input\_dic.keys()
				\end{flushleft}
			};
			%
			\node (pro2b) [process, below of=in2, xshift=5cm,yshift = -2cm] 
			{
				\begin{flushleft}
					chosen\_class\_key = \textbf{get\_text()}
					\\
					ChosenClass = print\_dic[chosen\_class\_key]
					\\
					new\_list\_keys = ChosenClass.inner.keys()
					\\
					update\_input\_dic(input\_dic,list\_keys[0],chosen\_class\_key,new\_list\_keys)
					\\
					input\_dic = HeadClass.inner()
					\\
					if check\_completeness(input\_dic)
					\\
					\quad print  HeadClass.parameters()
					\\
					else
					\\
					\quad first\_key = list\_keys[0]
					\\
					\quad list\_keys = list\_keys[1:]
					\\
					\quad for i in new\_list\_keys:
					\\
					\quad \quad list\_keys.append(0,first\_key+new\_list\_keys[i])
				\end{flushleft}		
			};
			%
			\node (pro3) [process, text width=9cm, below of=in2, yshift=-6cm] 
			{
				\begin{flushleft}
					print\_dic = HeadClass.dic\_to\_print(input\_dic,list\_keys[0]) 
					\\
					print input\_dic.keys()
				\end{flushleft}
			};
			
			\draw [arrow] (start) -- (in1);
			\draw [arrow] (in1) -- (pro1);
			\draw [arrow] (pro1) -- (in2);
			\draw [arrow] (in2) -- node[anchor=east] {head\_class\_set = \textbf{F}} (pro2a);
			\draw [arrow] (in2) -- node[anchor=south] {head\_class\_set = \textbf{T}} (pro2b);
			\draw [arrow] (pro2b) -- (pro3);
			\draw [arrow] (pro2a) -- (pro3);
	
		\end{tikzpicture}
	\end{figure}
	%
\end{landscape}

Example:
{\small
\begin{flushleft}
	\textbf{dic} = \{'Room1':Room1Class,'Room1':Room2Class,'Room3':Room3Class\}
	\\
	\# chosen head class (via user input)
	\\
	Room1Class.inner = \{'bed':bed\_dic,'lamp':lamp\_dic\}
	\\
	\# (incomplete)
	\\
	input\_dic = \{'bed':[],'Lamp':[]\}
	\\
	\# (incomplete)
	\\
	input\_dic = \{'bed':['Bed1':\{'pillow':[],'cushion':[]\}],'Lamp':['Lamp1':\{\}]\}	
	\\
	\# (complete) 
	\\
	input\_dic = \{'bed':['Bed1':\{'pillow':['Pillow1':\{\}],'cushion':['Cushion1':\{\}]\}],'Lamp':['Lamp1':\{\}]\}
	\\
	\# show parameters list
	\\
	Room1Class.to\_string(input\_dic)
\end{flushleft}
}